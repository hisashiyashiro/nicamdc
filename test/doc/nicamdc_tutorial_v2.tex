\documentclass[a4paper]{article}
%##########################################################################################
\title{{\Huge Tutorial of NICAM-DC for DCMIP2016\\
              on Yellowstone \\
        \vspace{1cm}{\Large Version 2.0} }}
\vspace{3cm}
\author{\Large NICAM group}
\vspace{3cm}
\date{\today}
\vspace{3cm}

\usepackage[dvipdfmx]{graphicx}
%\usepackage{amsmath}
%\usepackage{ascmac}
%\usepackage[round]{natbib}
\usepackage{tabularx}
\usepackage{color}
\usepackage{colortbl}
%\usepackage{fancybox}
\usepackage{url}
\usepackage[top=30mm,bottom=35mm,left=30mm,right=30mm]{geometry}

\begin{document}
\maketitle

\vspace{5cm}

\section*{Note}
%##########################################################################################
 This document is a tutorial of NICAM-DC for DCMIP2016. Procedures for three
 ideal experiments are explained assuming the Yellowstone environment.
 NICAM-DC is the dynamical core package, which is a part of NICAM full model.
 The development of NICAM with full physics has been co-developed mainly by
 the Japan Agency for Marine-Earth Science and Technology (JAMSTEC), Atmosphere
 and Ocean Research Institute (AORI) at The University of Tokyo, and RIKEN / Advanced
 Institute for Computational Science (AICS). A reference paper for NICAM is
 Tomita and Satoh (2004), Satoh et al. (2008). See also NICAM.jp (\url{http://nicam.jp/}).
 This distribution package (NICAM-DC) is released for the purpose of widespread
 use of NICAM. The license is according to BSD 2 Clause. Have lots of fun!

 \hrulefill
 \begin{itemize}
   \item Explanations are based on bash environment below.
         In the tcsh environment, use "setenv" command instead of "export".
         (e.g. \verb|> setenv NICAM_SYS "Yellowstone"|)
   \item A symbol of "\verb|>|" means execution of commands in the console.
   \item Gothic means output from standard output.
   \item \$\{TOP\}   means \verb|/glade/[your-working-dir]/nicamdc|
   \item \$\{SHARE\} means \verb|/glade/p/work/ryuji/public|
 \end{itemize}



\clearpage

\tableofcontents

\clearpage


\section{Quick Start}
%##########################################################################################

\subsection{Extend a tar file}
%------------------------------------------------------------------------------
 Copy a tar file to the disc space of scratch or work.

 \begin{verbatim}
   > cp nicamdc.20160606.tgz /grade/[your-working-dir]/
   > cd /grade/[your-working-dir]/
   > tar zxvf nicamdc.20160606.tgz
 \end{verbatim}

 \noindent By this command, a new directly of nicamdc is created,
 and contents are extended into the directly.

 \begin{itemize}
   \item[*] The tar file name would be changed by source code update.
 \end{itemize}


\subsection{Preparing Environment}
%------------------------------------------------------------------------------
 Set machine environmental parameters.
 \begin{verbatim}
   > module load ncl
   > module load cdo/1.6.3
   > module load mkl/10.3.11
 \end{verbatim}

 \begin{itemize}
   \item[*] You can check the loaded libraries by "module list" command.
   \item[*] Intel compiler, MPI, and netcdf are loaded as default sets.
 \end{itemize}

 \noindent Set NICAM environmental parameters.
 \begin{verbatim}
   > export NICAM_SYS="Yellowstone"
   > export ENABLE_NETCDF="T"
 \end{verbatim}

 \begin{itemize}
   \item[*] NICAM setting parameters are described in Makedef file,
    which is stored in \verb|${TOP}/sysdep|. The \verb|NICAM_SYS| parameter
    specifies the Makedef file.
   \item[*] In other machines, the parameter \verb|NICAM_SYS| should be changed
    to suitable one, for example "\verb|Linux64-gnu-ompi|" in GNU compiler
    with openMPI on Linux x86-64.
 \end{itemize}

 \noindent \textcolor{blue}{{\sf Recommendation}} \\
 To keep above settings, edit ".tcshrc" or ".bashrc".


\subsection{Compile}
%------------------------------------------------------------------------------
 \begin{table}[b]
 \begin{center}
 \caption{Corresponding test cases}
 \begin{tabularx}{150mm}{|l|X|} \hline
 \rowcolor[gray]{0.9} teat case name in NICAM & test case type \\ \hline
  DCMIP2016-11 & moist baroclinic wave test (161)       \\ \hline
  DCMIP2016-12 & idealized tropical cyclone test (162)  \\ \hline
  DCMIP2016-13 & supercell test (163)                   \\ \hline
 \end{tabularx}
 \end{center}
 \end{table}

 Change directly to test case directly, for example,
 \begin{verbatim}
   > cd ${TOP}/test/case/DCMIP2016-11
 \end{verbatim}

 \noindent In the \${TOP}/test/case, a lot of test cases sets are prepared.
 In those, the cases for DCMIP2016 are shown in Table 1.

 \noindent Compile the program using make command.
 \begin{verbatim}
   > make -j 4
 \end{verbatim}
 When the compile is finished correctly, a "\verb|nhm_driver|" is created
 in the current directly, which is an executable binary of NICAM.

 \begin{itemize}
   \item[*] The number of -j option is a number of parallel compile processes.
    To reduce elapsed time of compile, you can specify the number
    as more than two. We recommend 2 ~ 8 for the -j option.
 \end{itemize}




\subsection{Run experiments}
%------------------------------------------------------------------------------
   To run the model, type "make run", and then the job script is
   displayed in standard output. Hit "q" to quit.
 \begin{verbatim}
   > make run
   #! /bin/bash -x
   ##################################################################
   #
   # for NCAR Yellowstone (IBM iDataPlex Sandybridge)
   #
   ##################################################################
   #BSUB -a poe                  # set parallel operating environment
   #BSUB -P SCIS0006             # project code
   #BSUB -J nicamdc              # job name
   #BSUB -W 00:10                # wall-clock time (hrs:mins)
   #BSUB -n 10                   # number of tasks in job
   #BSUB -R "span[ptile=10]"     # run four MPI tasks per node
   #BSUB -q regular              # queue
   #BSUB -e errors.%J.nicamdc    # error file name
   #BSUB -o output.%J.nicamdc    # output file name
 \end{verbatim}

 \noindent If the job script is OK, submit a job to the machine.
 \begin{verbatim}
  > bsub < run.sh
 \end{verbatim}
 \noindent \textcolor{blue}{{\sf Caution}} : Do not miss the symbol "\verb|<|". \\

 \noindent You can check the status of your jobs by "bjobs" command.
 \begin{verbatim}
   > bjobs
   JOBID  USER  STAT QUEUE    FROM_HOST   EXEC_HOST   JOB_NAME SUBMIT_TIME
   159420 ryuji RUN  regular  yslogin3-ib 10*ys0531-i nicamdc  Jun  7 02:24
 \end{verbatim}

 \begin{itemize}
   \item[*] To see detail status, do "bjobs -l".
   \item[*] For detail, see \url{https://www2.cisl.ucar.edu/resources/
   computational-systems/yellowstone/using-computing-resources/
   running-jobs/platform-lsf-job-script-examples}
 \end{itemize}


\subsection{Post process}
%------------------------------------------------------------------------------
 After finish of test run, create the lat-lon grid data from
 the original icosahedral grid data.
 Before submit a job of post process, edit \verb|ico2ll_netcdf.sh|
 following your experimental settings.
 \begin{verbatim}
   > vi ico2ll_netcdf.sh

   [at Line 22]
   # User Settings
   # ---------------------------------------------------------------------

   glev=5          # g-level of original grid
   case=161        # test case number
   out_intev='day' # output interval (format: "1hr", "6hr", "day", "100s")
 \end{verbatim}

 \noindent If the job script is OK, submit a job to the machine.
 \begin{verbatim}
   > bsub < ico2ll_netcdf.sh
 \end{verbatim}

 \noindent The netcdf format data such as "\verb|nicam.161.200.L30.interp_latlon.nc|"
 is created by an "ico2ll" post-process program.


\subsection{Ploting}
%------------------------------------------------------------------------------
 In the DCMIP2016, NCL scripts are preparing to plot results
 for each experiments. The script use as below.
 \begin{verbatim}
   > ncl < ncl_script.161
 \end{verbatim}
 \noindent \textcolor{blue}{{\sf Caution}} : Do not miss the symbol "\verb|<|". \\

 \begin{itemize}
   \item[*] "ncview" and "grads" is also available to quick check.
   \item[*] NCL scripts (only given by LOC) are in \$\{SHARE\}/plots.
 \end{itemize}



\clearpage

\section{Change Model Configurations}
%##########################################################################################
 {\sf Terms}
 \begin{itemize}
   \item g-level (grid level): number of grid level, this is a number of
         subdivision times from the original icosahedron.
   \item r-level (region level): level of management groups. When r-level = 0,
         the management groups are ten groups. So, the number of
         available maximum MPI processes is ten.
 \end{itemize}



\subsection{Change Test Case}
%------------------------------------------------------------------------------
 \noindent In this section, configurations are explained here to run
 three experiments in DCMIP2016. NICAM-DC has already configuration sets
 for three test cases in the directory of \$\{TOP\}/test/case.
 At the first step, use these sets.

\subsubsection{preparing directory}
%------------------------------------------------------------------------------
 Change to the directory of target case. If you want to run test case 162,
 change to \${TOP}/test/case/DCMIP2016-12/. After that, make a directory.
 The directory of gl05rl00z30pe40 already may exists, we assume create it newly.
 \begin{verbatim}
    > mkdir gl05rl00z30pe40
    > cd gl05rl00z30pe40/
 \end{verbatim}

 \noindent Copy Makefile and configuration file from another directory
 of DCMIP2016 to the new direcotory, for example DCMIP2016-11.
 \begin{verbatim}
    > cp ../../DCMIP2016-11/gl05rl00z30pe10/Makefile ./
    > cp ../../DCMIP2016-11/gl05rl00z30pe10/nhm_driver.cnf ./
 \end{verbatim}

\subsubsection{Edit configuration file: nhm\_driver.cnf}
%------------------------------------------------------------------------------

 %-------------------------------
 \vspace{0.5cm}
 \noindent {\large{\sf edit for test case 161: moist baroclinic wave}}

 (symbols "\verb|<--|" means changed parameters)
 \begin{verbatim}
    > vi nhm_driver.cnf
     * snip *

     &RUNCONFPARAM
       NDIFF_LOCATION = 'IN_LARGE_STEP2',
       THUBURN_LIM    = .true.,
       RAIN_TYPE      = "WARM",
       AF_TYPE        = 'DCMIP',
      /

     * snip *

     &DYCORETESTPARAM
       init_type   = 'Jablonowski-Moist',    <--
       test_case   = '1',                    <--
       chemtracer  = .true.,                 <--
       prs_rebuild = .false.,
      /

     * snip *

     &FORCING_DCMIP_PARAM
       SET_DCMIP2016_11 = .true.,            <--
      /

     * snip *
 \end{verbatim}


 \noindent \textcolor{blue}{{\sf Note}}
 \begin{itemize}
   \item "init\_type" should be specified as "Jablonowski-Moist".
   \item "test\_case" can be choose from 1 ~ 6.\\
          case 1: perturbation: exponential / with moisture \\
          case 2: perturbation: stream function / with moisture \\
          case 3: perturbation: exponential / without moisture \\
          case 4: perturbation: stream function / without moisture \\
          case 5: no perturbation / with moisture \\
          case 6: no perturbation / without moisture
   \item \verb|FORCING_DCMIP_PARAM| should be specified as "\verb|SET_DCMIP2016_11 = .true.|".
   \item "step" in \verb|NMHISD| should be changed following required history output interval
           as described in DCMIP2016 Test Case Document.
   \item items of history output variables, which specified by "NMHIST", should be added
         following the requirement in DCMIP2016 Test Case Document.
   \item "small\_planet\_factor" in CNSTPARAM should be set as 1.
 \end{itemize}

 %-------------------------------
 \vspace{0.5cm}
 \noindent {\large {\sf edit for test case 162: ideal tropical cyclone}}

 (symbols "\verb|<--|" means changed parameters)
 \begin{verbatim}
    > vi nhm_driver.cnf
     * snip *

     &RUNCONFPARAM
       NDIFF_LOCATION = 'IN_LARGE_STEP2',
       THUBURN_LIM    = .true.,
       RAIN_TYPE      = "WARM",
       AF_TYPE        = 'DCMIP',
      /

     * snip *

     &DYCORETESTPARAM
       init_type   = 'Tropical-Cyclone',     <--
      /

     * snip *

     &FORCING_DCMIP_PARAM
       SET_DCMIP2016_12 = .true.,            <--
      /

     * snip *
 \end{verbatim}

 \noindent \textcolor{blue}{{\sf Note}}
 \begin{itemize}
   \item "init\_type" should be specified as "Tropical-Cyclone".
   \item \verb|FORCING_DCMIP_PARAM| should be specified as "\verb|SET_DCMIP2016_12 = .true.|".
   \item "step" in \verb|NMHISD| should be changed following required history output interval
           as described in DCMIP2016 Test Case Document.
   \item items of history output variables, which specified by "NMHIST", should be added
         following the requirement in DCMIP2016 Test Case Document.
   \item "small\_planet\_factor" in CNSTPARAM should be set as 1.
 \end{itemize}


  %-------------------------------
 \vspace{0.5cm}
 \noindent {\large{\sf edit for test case 163: supercell}}

 (symbols "\verb|<--|" means changed parameters)
 \begin{verbatim}
    > vi nhm_driver.cnf
     * snip *

     &RUNCONFPARAM
       NDIFF_LOCATION = 'IN_LARGE_STEP2',
       THUBURN_LIM    = .true.,
       RAIN_TYPE      = "WARM",
       AF_TYPE        = 'DCMIP',
      /

     * snip *

     &DYCORETESTPARAM
       init_type   = 'Supercell',            <--
       test_case  = '1',                     <--
      /

     * snip *

     &FORCING_DCMIP_PARAM
       SET_DCMIP2016_13 = .true.,            <--
      /

     * snip *
 \end{verbatim}

 \noindent \textcolor{blue}{{\sf Note}}
 \begin{itemize}
   \item "init\_type" should be specified as "Supercell".
   \item "test\_case" can be choose from 1 ~ 6.\\
          case 1: with initial perturbation \\
          case 2: without initial perturbation
   \item \verb|FORCING_DCMIP_PARAM| should be specified as "\verb|SET_DCMIP2016_13 = .true.|".
   \item "step" in \verb|NMHISD| should be changed following required history output interval
           as described in DCMIP2016 Test Case Document.
   \item items of history output variables, which specified by "NMHIST", should be added
         following the requirement in DCMIP2016 Test Case Document.
   \item \textcolor{red}{"small\_planet\_factor" in CNSTPARAM should be set as 120}.
   \item \textcolor{red}{"earth\_angvel" in CNSTPARAM should be set as 0}.
 \end{itemize}

 \noindent After above edit, you can run the experiment
 by the same manner in Section 1.4.
 \begin{verbatim}
    > make run
    > bsub < run.sh
 \end{verbatim}


\subsection{Change Physics Schemes}
%------------------------------------------------------------------------------
 \noindent Default settings for each test cases in DCMIP2016 is set
 in the pre-existing configuration file. You can change these settings
 as you like. Note that we have not yet checked all the combinations of
 physics schemes for all test cases. \\


 \noindent {\large{\sf use Large scale condensation instead of kessler}}

 \noindent The default setting for cloud microphysics is Kessler scheme.
 To use Large scale condensation (Reed and Jablonowski (2012) precip scheme),
 add "\verb|SET_DCMIP2016_LSC|" with true sign. An example for test case 161
 is shown below.

 (symbols "\verb|<--|" means changed parameters)
 \begin{verbatim}
    > vi nhm_driver.cnf
     * snip *

     &FORCING_DCMIP_PARAM
       SET_DCMIP2016_11 = .true.,
       SET_DCMIP2016_LSC = .true.,      <--
      /

     * snip *
 \end{verbatim}


 \noindent {\large{\sf no cloud physics}}

 \noindent To run without any cloud physics, add "\verb|SET_DCMIP2016_DRY|" with true sign.
 An example for test case 161 is shown below.

 (symbols "\verb|<--|" means changed parameters)
 \begin{verbatim}
    > vi nhm_driver.cnf
     * snip *

     &FORCING_DCMIP_PARAM
       SET_DCMIP2016_11 = .true.,
       SET_DCMIP2016_DRY = .true.,      <--
      /

     * snip *
 \end{verbatim}


 \noindent {\large{\sf use George Bryan PBL}}

 \noindent The default setting for PBL scheme is Reed and Jablonowski (2012).
 To use George Bryan PBL, add "\verb|SM_PBL_Bryan|" with true sign.
 This option is available only for Tropical cyclone case (162).
 An example is shown below.

 (symbols "\verb|<--|" means changed parameters)
 \begin{verbatim}
    > vi nhm_driver.cnf
     * snip *

     &FORCING_DCMIP_PARAM
       SET_DCMIP2016_12 = .true.,
       SM_PBL_Bryan     = .true.,      <--
      /

     * snip *
 \end{verbatim}


 \noindent {\large{\sf no physics}}

 \noindent To run any physics scheme, specify "NONE" to the parameter
 of AF\_TYPE in RUNCONFPARAM.
 An example for test case 161 is shown below.

 (symbols "\verb|<--|" means changed parameters)
 \begin{verbatim}
    > vi nhm_driver.cnf
     * snip *

     &RUNCONFPARAM
       NDIFF_LOCATION = 'IN_LARGE_STEP2',
       THUBURN_LIM    = .true.,
       RAIN_TYPE      = "WARM",
       AF_TYPE        = 'NONE',       <--
      /

     * snip *
 \end{verbatim}



\subsection{Increase MPI processes}
%------------------------------------------------------------------------------
 \noindent To reduce elasped time of the model execution, we can increase
 number of MPI processes. For example, edit to change to use 40 MPI processes
 with g-level 5 in test case 161.

 To increase MPI processes up to 40, r-level should be rised from 0 to 1
 because the upper limit of processes in r-level 0 is 10 processes.

\subsubsection{preparing directory}
%------------------------------------------------------------------------------
 We assume in \${TOP}/test/case/DCMIP2016-11/
 \begin{verbatim}
    > mkdir gl05rl01z30pe40    <-- r-level is 1
    > cd gl05rl01z30pe40/
 \end{verbatim}

 \noindent Copy Makefile and configuration file to new direcotory.
 \begin{verbatim}
    > cp ../gl05rl00z30pe10/Makefile ./
    > cp ../gl05rl00z30pe10/nhm_driver.cnf ./
 \end{verbatim}

\subsubsection{Edit Makefile}
%------------------------------------------------------------------------------
 (symbols "\verb|<--|" means changed parameters) \\
 On the Lines from 17 to 21, edit parameters.
 \begin{verbatim}
    > vi Makefile
     glevel = 5
     rlevel = 1      <--
     nmpi   = 40     <--
     zlayer = 30
     vgrid  = vgrid30_stretch_30km_dcmip2016.dat
 \end{verbatim}

\subsubsection{Edit configuration file: nhm\_driver.cnf}
%------------------------------------------------------------------------------
 (symbols "\verb|<--|" means changed parameters)
 \begin{verbatim}
    > vi nhm_driver.cnf
     * snip *
     &ADMPARAM
       glevel      = 5,
       rlevel      = 1,                     <--
       vlayer      = 30,
       rgnmngfname = "rl01-prc40.info",     <--
      /

     &GRDPARAM
       hgrid_io_mode = "ADVANCED",
       hgrid_fname   = "boundary_GL05RL01",  <--
       VGRID_fname   = "vgrid30_stretch_30km_dcmip2016.dat",
       vgrid_scheme  = "LINEAR",
       topo_fname    = "NONE",
      /

     * snip *

     &RESTARTPARAM
       input_io_mode     = 'IDEAL',
       output_io_mode    = 'ADVANCED',
       output_basename   = 'restart_all_GL05RL01z30', <--
       restart_layername = 'ZSALL32_DCMIP16',
      /
 \end{verbatim}

 \noindent After above edit, you can run the experiment
 by the same manner in Section 1.4.
 \begin{verbatim}
    > make run
    > bsub < run.sh
 \end{verbatim}


\subsection{Change grid spacing}
%------------------------------------------------------------------------------
 \noindent This is an example to change grid spacing of g-level 6
 (approxi. 120 km) with 40 MPI processes in test case 161.
 When horizontal grid space is changed, some additional settings
 should be changed, for example, interval of time integration (DTL),
 maximum number of time steps (LSTEP\_MAX), numerical filter parameters,
 and output interval of history data.

\subsubsection{preparing directory}
%------------------------------------------------------------------------------
 We assume in \${TOP}/test/case/DCMIP2016-11/
 \begin{verbatim}
    > mkdir gl06rl01z30pe40  <-- g-level is 6, and r-level is 1
    > cd gl06rl01z30pe40/
 \end{verbatim}

 \noindent Copy Makefile and configuration file to new direcotory.
 \begin{verbatim}
    > cp ../gl05rl00z30pe10/Makefile ./
    > cp ../gl05rl00z30pe10/nhm_driver.cnf ./
 \end{verbatim}

\subsubsection{Edit Makefile}
%------------------------------------------------------------------------------
 (symbols "\verb|<--|" means changed parameters) \\
 On the Lines from 17 to 21, edit parameters.
 \begin{verbatim}
    > vi Makefile
     glevel = 6      <--
     rlevel = 1      <--
     nmpi   = 40     <--
     zlayer = 30
     vgrid  = vgrid30_stretch_30km_dcmip2016.dat
 \end{verbatim}

\subsubsection{Edit configuration file: nhm\_driver.cnf}
%------------------------------------------------------------------------------
 \noindent A guideline of changing interval of time integration (DTL) is \\
 {\sf take 1/2 of DTL by one up of g-level}.

 \noindent A guideline of changing numerical filter parameters is \\
 {\sf take 1/8 of coefficient value by one up of g-level}. \\

 \noindent (symbols "\verb|<--|" means changed parameters)
 \begin{verbatim}
    > vi nhm_driver.cnf
     * snip *
     &ADMPARAM
       glevel      = 6,                     <--
       rlevel      = 1,                     <--
       vlayer      = 30,
       rgnmngfname = "rl01-prc40.info",     <--
      /

     &GRDPARAM
       hgrid_io_mode = "ADVANCED",
       hgrid_fname   = "boundary_GL06RL01",  <--
       VGRID_fname   = "vgrid30_stretch_30km_dcmip2016.dat",
       vgrid_scheme  = "LINEAR",
       topo_fname    = "NONE",
      /

     &TIMEPARAM
       DTL         = 300.D0,     <--
       INTEG_TYPE  = "RK3",
       LSTEP_MAX   = 4320,       <--
       start_date  = 0000,1,1,0,0,0
      /

     * snip *

     &RESTARTPARAM
       input_io_mode     = 'IDEAL',
       output_io_mode    = 'ADVANCED',
       output_basename   = 'restart_all_GL06RL01z30', <--
       restart_layername = 'ZSALL32_DCMIP16',
      /

     * snip *

     &NUMFILTERPARAM
       lap_order_hdiff   = 2,
       hdiff_type        = 'NONLINEAR1',
       Kh_coef_maxlim    = 1.500D+16,    <--
       Kh_coef_minlim    = 1.500D+15,    <--
       ZD_hdiff_nl       = 20000.D0,
       divdamp_type      = 'DIRECT',
       lap_order_divdamp = 2,
       alpha_d           = 1.50D15,      <--
       gamma_h_lap1      = 0.0D0,
       ZD                = 40000.D0,
       alpha_r           = 0.0D0,
      /

     * snip *

     &NMHISD
       output_io_mode   = 'ADVANCED' ,
       histall_fname    = 'history'  ,
       hist3D_layername = 'ZSDEF30_DCMIP16',
       NO_VINTRPL       = .false.    ,
       output_type      = 'SNAPSHOT' ,
       step             = 288        ,    <--
       doout_step0      = .true.     ,
      /
 \end{verbatim}

 \noindent After above edit, you can run the experiment
 by the same manner in Section 1.4.
 \begin{verbatim}
    > make run
    > bsub < run.sh
 \end{verbatim}


\clearpage

\section{Requested Test Cases in DCMIP2016}
%##########################################################################################
 The requested test cases are summarized in this section.
 Take care to test settings, because a part of test cases is not described
 in the Test Case Document.


\subsection{Moist Baroclinic Wave}
%------------------------------------------------------------------------------
In the moist baroclinic wave, two types of tests are requested.

\subsubsection{common settings}
 \begin{itemize}
   \item default horizontal resolution: 1 degree.
   \item vertical grid arrangement: 30 levels (stretched)
   with 120m resolution in the lowest model level.
   The model cap should be between 30km and 50km.
 \end{itemize}

\subsubsection{case: 161}
 \begin{itemize}
   \item employ Kessler precipitation
   \item employ surface fluxes (test = 1)
   \item employ boundary layer
   \item no RJ2012\_precip
   \item employ Terminator physics
 \end{itemize}

\subsubsection{case: 161-preciponly}
 \begin{itemize}
   \item employ Kessler precipitation
   \item no surface fluxes
   \item no boundary layer
   \item no RJ2012\_precip
   \item employ Terminator physics
 \end{itemize}

\subsubsection{output variables}
 \begin{itemize}
   \item daily frequency
   \item 3D param: Qv, Qc, Qr, U, V, W, Theta, T, P (or Rho)
   \item 2D param: surface pressure, instantaneous precip 6-hourly averaged precip.
   \item Additional 2D diagnostics are required for the Terminator test,
         as described in the test case document.
 \end{itemize}


\subsection{Tropical Cyclone}
%------------------------------------------------------------------------------
In the tropical cyclone, two types of tests are requested.

\subsubsection{common settings}
 \begin{itemize}
   \item employ Kessler precipitation
   \item employ Reed-Jablonowski simple physics enable
   \item employ surface fluxes (test = 1)
   \item no RJ2012\_precip
   \item vertical levels setting is same with the baroclinic wave test.
   \item required horiz. grid spacing = 1.0 and 0.5 deg.
         (optionally 0.25 degree)
 \end{itemize}

\subsubsection{case: 162-rjpbl}
 \begin{itemize}
   \item employ RJ2012 PBL (default PBL in simple physics)
 \end{itemize}

\subsubsection{case: 162-bryanpbl}
 \begin{itemize}
   \item employ Bryan TC PBL
 \end{itemize}

\subsubsection{output variables}
 \begin{itemize}
   \item 6 hours interval
   \item default horiz. grid space: 1 degree (encourage groups to try a 0.25).
   \item 3D param: Qv, Qc, Qr, U, V, W. Theta, T, P (or Rho).
   \item 2D param: surface pressure instantaneous 6-hourly,
             instantaneous precip 6-hourly averaged precip.
 \end{itemize}


\subsection{Supercell}
%------------------------------------------------------------------------------
\subsubsection{case: 163}
 \begin{itemize}
   \item employ Kessler physics
   \item no RJ2012\_precip
   \item no surface fluxes
   \item default resolution = 1 degree.
   \item vertical levels: 40 levs (~500m vertical grid spacing, uniformly spaced)
   \item model top: ~20 km
 \end{itemize}

\subsubsection{case: 162-bryanpbl}
 \begin{itemize}
   \item employ Bryan TC PBL
 \end{itemize}

\subsubsection{output variables}
 \begin{itemize}
   \item default output interval: 5 minutes
   \item 3D Qv, Qc, Qr, U, V, W, Theta, plus at least one of Phi, T, Tv, Rh*o or P*).
   \item 2D surface pressure and instantaneous and 5-minute averaged precipitation.
 \end{itemize}

\clearpage

\section{Appendix: Configuration Parameters}
%##########################################################################################

 columns of example show settings of test case 161 in g-level 5.

\begin{table}[htb]
\begin{center}
\caption{ADMPARAM (Model Administration Parameters)}
\begin{tabularx}{150mm}{|l|l|l|X|} \hline
 \rowcolor[gray]{0.9} parameters & example & kind & description          \\ \hline
 glevel      & 5                 & int  & number of g-level              \\ \hline
 rlevel      & 0                 & int  & number of r-level              \\ \hline
 vlayer      & 30                & int  & number of vertical layers      \\ \hline
 rgnmngfname & "rl00-prc10.info" & char & name of region management file \\ \hline
\end{tabularx}
\end{center}
\end{table}

\begin{table}[htb]
\begin{center}
\caption{GRDPARAM (Grid Setting Parameters)}
\begin{tabularx}{150mm}{|l|l|l|X|} \hline
 \rowcolor[gray]{0.9} parameters & example & kind & description      \\ \hline
 \verb|hgrid_io_mode| & "ADVANCED" & char & IO mode of horizontal grid file \\ \hline
 \verb|hgrid_fname|   & "\verb|boundary_GL05RL00|"                  & char & name of horizontal grid file \\ \hline
 \verb|VGRID_fname|   & "\verb|vgrid30_stretch_30km_dcmip2016.dat|" & char & name of vertical grid file \\ \hline
 \verb|vgrid_scheme|  & "LINEAR"   & char & IO mode of vertical grid file   \\ \hline
 \verb|topo_fname|    & "NONE"     & char & name of topography file         \\ \hline
\end{tabularx}
\end{center}
\end{table}

\begin{table}[htb]
\begin{center}
\caption{TIMEPARAM (Time Integration Setting Parameters)}
\begin{tabularx}{150mm}{|l|l|l|X|} \hline
 \rowcolor[gray]{0.9} parameters & example & kind & description          \\ \hline
 DTL        & 600.D0 & real & interval of time step (s) \\ \hline
 \verb|INTEG_TYPE| & "RK3"  & char & time integration scheme \\ \hline
 \verb|LSTEP_MAX|  & 2160   & int  &  time integration steps \\ \hline
 \verb|start_date| & 0000,1,1,0,0,0 & int (array) & date of initial time \\ \hline
\end{tabularx}
\end{center}
\end{table}

\begin{table}[htb]
\begin{center}
\caption{RUNCONFPARAM (Common Configurations)}
\begin{tabularx}{150mm}{|l|l|l|X|} \hline
 \rowcolor[gray]{0.9} parameters & example & kind & description          \\ \hline
 \verb|NDIFF_LOCATION|        & '\verb|IN_LARGE_STEP2|' & char  & setting of numerical diffusion \\ \hline
 \verb|THUBURN_LIM|           & .true.      & logical & use of the limiter \\ \hline
 \verb|EIN_TYPE|              & 'SIMPLE'    & char  & evaluation type of internal energy \\ \hline
 \verb|RAIN_TYPE|             & 'WARM'      & char & date of initial time \\ \hline
 \verb|CHEM_TYPE|             & 'PASSIVE'   & char & chemical tracer type \\ \hline
 \verb|AF_TYPE|               & 'DCMIP2016' & char & type of forcing (physics step) \\ \hline
\end{tabularx}
\end{center}
\end{table}

\begin{table}[htb]
\begin{center}
\caption{CHEMVARPARAM (Chemical Tracer Settings)}
\begin{tabularx}{150mm}{|l|l|l|X|} \hline
 \rowcolor[gray]{0.9} parameters & example & kind & description          \\ \hline
 \verb|CHEM_TRC_vmax| & 2 & int &  maximum number of chemical tracers \\ \hline
\end{tabularx}
\end{center}
\end{table}

\begin{table}[htb]
\begin{center}
\caption{BSSTATEPARAM (Basic (reference) State Parameters)}
\begin{tabularx}{150mm}{|l|l|l|X|} \hline
 \rowcolor[gray]{0.9} parameters & example & kind & description          \\ \hline
 \verb|ref_type| & 'NOBASE' & char & type of reference state \\ \hline
\end{tabularx}
\end{center}
\end{table}

\begin{table}[htb]
\begin{center}
\caption{RESTARTPARAM (Initialize/Restart Setting Parameters)}
\begin{tabularx}{150mm}{|l|l|l|X|} \hline
 \rowcolor[gray]{0.9} parameters & example & kind & description          \\ \hline
 \verb|input_io_mode|     & 'IDEAL'                   & char & IO mode of input file (for initialize) \\ \hline
 \verb|output_io_mode|    & 'ADVANCED'                & char & IO mode of output file (for restart) \\ \hline
 \verb|output_basename|   & '\verb|restart_all_GL05RL00z30|' & char & name of output file \\ \hline
 \verb|restart_layername| & '\verb|ZSALL32_DCMIP16|'         & char & name of vertical lev. info. file for restart \\ \hline
\end{tabularx}
\end{center}
\end{table}

\begin{table}[htb]
\begin{center}
\caption{DYCORETESTPARAM (Dynamical-core Test Parameters)}
\begin{tabularx}{150mm}{|l|l|l|X|} \hline
 \rowcolor[gray]{0.9} parameters & example & kind & description          \\ \hline
 \verb|init_type|    & 'Jablonowski-Moist' & char & test case name \\ \hline
 \verb|test_case|    & '1'     & char & test case number (not DCMIP test case number) \\ \hline
 chemtracer   & .true.  & logical & switch of chemical tracer \\ \hline
 \verb|prs_rebuild|  & .false. & logical & switch of initial pressure re-calculation\\ \hline
\end{tabularx}
\end{center}
\end{table}

\begin{table}[htb]
\begin{center}
\caption{FORCING\_PARAM (Forcing (Physics) Setting Parameters)}
\begin{tabularx}{150mm}{|l|l|l|X|} \hline
 \rowcolor[gray]{0.9} parameters & example & kind & description          \\ \hline
 \verb|NEGATIVE_FIXER|  & .true.  & logical & switch of negative fixer \\ \hline
 \verb|UPDATE_TOT_DENS| & .false. & logical & switch of total density update \\ \hline
\end{tabularx}
\end{center}
\end{table}

\begin{table}[htb]
\begin{center}
\caption{FORCING\_DCMIP\_PARAM (DCMIP2016 Physics Setting)}
\begin{tabularx}{150mm}{|l|l|l|X|} \hline
 \rowcolor[gray]{0.9} parameters & example & kind & description          \\ \hline
 \verb|SET_DCMIP2016_11| & .true.  & logical & physics set for test 161 (exclusive use) \\ \hline
 \verb|SET_DCMIP2016_12| & .false. & logical & physics set for test 162 (exclusive use) \\ \hline
 \verb|SET_DCMIP2016_13| & .false. & logical & physics set for test 163 (exclusive use) \\ \hline
\end{tabularx}
\end{center}
\end{table}

\begin{table}[htb]
\begin{center}
\caption{CNSTPARAM (Constant Parameters)}
\begin{tabularx}{150mm}{|l|l|l|X|} \hline
 \rowcolor[gray]{0.9} parameters & default & kind & description          \\ \hline
 \verb|earth_radius| & 6.37122D+6  & real & radius of the earth (m) \\ \hline
 \verb|earth_angvel| & 7.292D-5    & real & angular velocity of the earth (s-1) \\ \hline
 \verb|small_planet_factor| & 1.D0 & real & small planet facter (X) \\ \hline
 \verb|earth_gravity|       & 9.80616D0 & real & gravity acceleration (m s-2) \\ \hline
 \verb|gas_cnst|            & 287.0D0   & real & ideal gas constant for dry air (J kg-1 K-1) \\ \hline
 \verb|specific_heat_pre|   & 1004.5D0  & real & specific heat capacity at constant pressure (J kg-1 K-1) \\ \hline
\end{tabularx}
\end{center}
\end{table}

\begin{table}[htb]
\begin{center}
\caption{NUMFILTERPARAM (Numerical Filter Settings)}
\begin{tabularx}{150mm}{|l|l|l|X|} \hline
 \rowcolor[gray]{0.9} parameters & default & kind & description          \\ \hline
 \verb|lap_order_hdiff| & 2            & int  & order of horizontal diffusion \\ \hline
 \verb|hdiff_type|      & 'NONLINEAR1' & char & horizontal diffusion type \\ \hline
 \verb|Kh_coef_maxlim|  & 1.200D+17    & real & maximum limit of Kh coefficient (for Non-Linear) \\ \hline
 \verb|Kh_coef_minlim|  & 1.200D+16    & real & minimum limit of Kh coefficient (for Non-Linear) \\ \hline
 \verb|ZD_hdiff_nl|     & 20000.D0     & real & effective bottom level of horiz. diff. (for Non-Linear) \\ \hline
 \verb|divdamp_type|    &  'DIRECT'    & char & divergence dumping type \\ \hline
 \verb|lap_order_divdamp| & 2          & int  & order of divergence dumping \\ \hline
 \verb|alpha_d|         & 1.20D16      & real & specific value of coefficient for divergence dumping \\ \hline
\end{tabularx}
\end{center}
\end{table}

\begin{table}[htb]
\begin{center}
\caption{EMBUDGETPARAM (Budget Monitoring Parameters)}
\begin{tabularx}{150mm}{|l|l|l|X|} \hline
 \rowcolor[gray]{0.9} parameters & example & kind & description          \\ \hline
 \verb|MNT_ON|   & .true. & logical & switch of monitoring \\ \hline
 \verb|MNT_INTV| & 72     & int     & monitoring interval (steps) \\ \hline
\end{tabularx}
\end{center}
\end{table}

\begin{table}[htb]
\begin{center}
\caption{NMHISD (Common History Output Parameters)}
\begin{tabularx}{150mm}{|l|l|l|X|} \hline
 \rowcolor[gray]{0.9} parameters & default & kind & description          \\ \hline
 \verb|output_io_mode|   & 'ADVANCED' & char    & IO mode of history output file \\ \hline
 \verb|histall_fname|    & 'history'  & char    & name of history output file \\ \hline
 \verb|hist3D_layername| & '\verb|ZSDEF30_DCMIP16|' & char & name of vertical lev. info. file for history \\ \hline
 \verb|NO_VINTRPL|       & .false.    & logical & switch of vertical interpolation \\ \hline
 \verb|output_type|      & 'SNAPSHOT' & char    & output value type (snapshot or average) \\ \hline
 step             & 72         & int     & output interval (steps) \\ \hline
 \verb|doout_step0|      & .true.     & logical & switch of output initial condition \\ \hline
\end{tabularx}
\end{center}
\end{table}

\begin{table}[t]
\begin{center}
\caption{NMHIST (Settings of History Output Items)}
\begin{tabularx}{150mm}{|l|l|l|X|} \hline
 \rowcolor[gray]{0.9} parameters & example & kind & description          \\ \hline
 item  & '\verb|ml_u|' & char & name of output variable in the model \\ \hline
 file  & 'u'    & char & name of output variable in the file \\ \hline
 ktype & '3D'   & char & dimension type of output variable \\ \hline
\end{tabularx}
\end{center}
\end{table}


%##########################################################################################
\end{document}
